\documentclass{article}
\usepackage[utf8]{inputenc}
\usepackage{amsmath}
\usepackage{graphicx}
\usepackage{float}
\graphicspath{  {./Figures/}  } 

% ========== Title Page ==========
\title{Heat Pump}
\author{Henry V. Gilbert \\
	The University of Tennessee Knoxville \\
	960 Riverside Forest Way
	Apt. 002 \\
	Knoxville TN, 37915
	}
\date {March 2 2019}

\begin{document}
\maketitle



% ========== Intro Section  ==========
\section{Introduction}
\paragraph{Background} 
The vaopr-compression refrigeration cycle has been considered a modern masterpiece of engineering. Having the ability
to manipulate properties based on simple engineering principles has had insurmountable implications in the industry. 
Found in engine cooling systems, HVAC systems for homes and buildings, and even computer cooling, this work of 
art is the most ... 

	A heat pump is a device whose main objective is to provide a temperature gradient for the exchange of heat within
an evaporator. Internally, the temperatures and pressures of a refrigerant are manipulated to setup the heat exchangers. 
	
	At the most fundamental level, the heat pump operates on three main principles. 1. Compressing a gas increases its temperature, and expanding a gas will lower it. This is due to the intermolecular forces having less room to move around, thus increasing kinetic energy and therefore increasing temperature. 2. Heat will naturally flow from a hotter substance to a lower substance. These two principles guide the vapor compression cycle. The idealized cycle consists of four fundamental components: compressor, condenser, throttle, and evaporator. Although a real system consists of several more valves and lines, these four components are enough to understand the most primitive ideas behind the cycle. Starting at the compressor, a refrigerant gas is compressed as a result of work input into the system. As a result, the gas, now compressed, as risen in pressure, and therefore temperature. Exiting the compressor as a superheated vapor. the refrigerant enters the condenser. At the condenser, the superheated gas exchanges heat with the outside air, thus lowering its temperature while still maintaining the high pressure. As a result of giving off heat, the once superheated refrigerant gas "condenses" and becomes a high pressure liquid. This high pressure liquid enters the throttle valve and experiences a significant drop in pressure. Similar to compression, where increasing pressure increases temperature, the reverse is true. Expanding the liquid reduces the pressure and therefore reduces the temperature. This subcooled liquid enters the evaporator, where heat is absorbed from the indoor unit and the customer experience the cool air. The heat from the inside of the room is absorbed by the subcooled liquid to the point where this liquid begins to evaporate into a low pressure gas, where it enters the compressor and completes the cycle. 
	
	A heat pump works the same way, but with a reversing valve. 
	
	
\paragraph{Objectives}

% ========== Methodology ==========
\section{Methodology}
\paragraph {Apparatus}
\paragraph{Test Procedure}
\paragraph{Data Analysis}


% ========== Results ==========
\section{Results and Discussion}\label{conclusions}
\paragraph {Results} 

% ========== Discussion section  ==========
\section {Conclusions and Recommendations}


% ========== References ==========
\section {References}


% ========== Appendices ==========
\section {Appendices}

\end{document}
